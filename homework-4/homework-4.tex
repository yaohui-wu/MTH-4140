\documentclass[12pt]{article}
\usepackage[letterpaper, margin=1in]{geometry}
\usepackage{mlmodern}
\usepackage{amsthm}
\usepackage{amssymb}

\theoremstyle{definition}
\newtheorem{problem}{Problem}
\newenvironment*{solution}{\begin{proof}[Solution]}{\end{proof}}
\renewcommand*{\qedsymbol}{\(\blacksquare\)}

\title{MTH 4140 Graph Theory}
\author{Yaohui Wu}
\date{May 5, 2024}

\begin{document}
\maketitle
\section*{Homework 4}
\section*{Introduction to Graph Theory}

\subsection*{Chapter 4 Connectivity and Paths}
\subsection*{Section 4.1 Cuts and Connectivity}

\begin{problem}
    Excercise 4.1.1
\end{problem}
\begin{solution}
    (a) A graph is \(k\)-connected if its connectivity is at least \(k\).
    If a graph \(G\) is 2-connected, then its connectivity is at least 2.
    Given \(G\) has connectivity 4 so we have \(\kappa(G)=4\geq2\) which is true.
    The statement ``Every graph with connectivity 4 is 2-connected." is true. \\
    (b) A 3-connected graph \(G\) has connectivity at least 3 so \(\kappa(G)\geq3\).
    Consider the graph \(K_5\), we know that \(\kappa(G)\leq\delta(G)\) so \(\kappa(K_5)\leq4\).
    We know that \(\kappa(K_n)=n-1\) so we get that \(\kappa(K_5)=4>3\).
    This is a counterexample since \(K_5\) is 3-connected but it has connectivity 4.
    The statement ``Every 3-connected graph has connectivity 3.'' is false. \\
    (c) A graph is \(k\)-edge-connected if every disconnecting set has at least \(k\) edges.
    Consider a \(k\)-connected graph \(G\), we know that \(\kappa(G)\leq\kappa'(G)\) from Whitney's theorem.
    We can deduce that \[k \leq \kappa(G) \leq \kappa'(G) \]
    The edge-connectivity \(\kappa'(G)\) is the minimum size of a disconnecting set and it is at least \(k\) so \(G\) is \(k\)-edge-connected.
    The statement ``Every \(k\)-connected graph is \(k\)-edge-connected.'' is true. \\
    (d) Consider a \(k\)-edge-connected graph \(G\) with connectivity \(\kappa(G)\) and \\ edge-connectivity \(\kappa'(G)\).
    We have \(\kappa(G)\leq\kappa'(G)\) and \(k\leq\kappa'(G)\).
    If \(k>\kappa(G)\), then we have \[\kappa(G)<k\leq\kappa'(G)\] and Whitney's theorem still holds so \(G\) is not \(k\)-connected. \\
    The statement ``Every \(k\)-edge-connected graph is \(k\)-connected.'' is false.
\end{solution}

\begin{problem}
    Excercise 4.1.8(b)
\end{problem}
\begin{solution}
    Let \(G\) be the graph on the right, we see that \(G\) is a 4-regular graph so the minimum degree of \(G\) is 4.
    Since every vertex is connected to four other vertices,
    we must remove 4 incidental edges of any vertex at minimum to disconnect the graph.
    Similarly, we must remove 4 adjacent vertices of any vertex at minimum to disconnect the graph.
    Therefore, the edge-connectivity of \(G\) is 4 and the connectivity of \(G\) is 4.
    The connectivity \(\kappa(G)\), edge-connectivity \(\kappa'(G)\), and the minimum degree \(\delta(G)\) are
    \[\kappa(G) = \kappa'(G) = \delta(G) = 4\] which follows immediately from Whitney's theorem.
\end{solution}

\begin{problem}
    Excercise 4.1.11
\end{problem}
\begin{solution}
    It is proved that \(\kappa'(G)=\kappa(G)\) when \(G\) is a simple graph with \(\Delta(G)\leq3\).
\end{solution}

\subsection*{Chapter 5 Coloring of Graphs}
\subsection*{Section 5.1 Vertex Coloring and Upper Bounds}
\subsection*{Section 5.2 Structure of \(k\)-chromatic Graphs}
\subsection*{Section 5.3 Enumerative Aspects}
\end{document}