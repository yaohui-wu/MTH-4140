\documentclass[12pt]{article}
\usepackage[letterpaper, margin=1in]{geometry}
\usepackage{mlmodern}
\usepackage{amsthm}
\usepackage{amssymb}

\theoremstyle{definition}
\newtheorem{problem}{Problem}
\newenvironment*{solution}{\begin{proof}[Solution]}{\end{proof}}
\renewcommand*{\qedsymbol}{\(\blacksquare\)}

\title{MTH 4140 Graph Theory}
\author{Yaohui Wu}
\date{May 2, 2024}

\begin{document}
\maketitle
\section*{Homework 4}
\section*{Introduction to Graph Theory}

\subsection*{Chapter 4 Connectivity and Paths}
\subsection*{Section 4.1 Cuts and Connectivity}
\begin{problem}
    Excercise 4.1.1
\end{problem}
\begin{solution}
    (a) We know that a graph is \(k\)-connected if its vertex connectivity is at least \(k\).
    If a graph \(G\) is 2-connected, then its vertex connectivity is at least 2.
    Given \(G\) has vertex connectivity 4 so we have \(\kappa(G)=4\geq2\) which is true.
    The statement ``Every graph with connectivity 4 is 2-connected." is true. \\
    (b) A 3-connected graph \(G\) has vertex connectivity at least 3 so \(\kappa(G)\geq3\).
    Consider the graph \(K_5\), we know that \(\kappa(G)\leq\delta(G)\) so \(\kappa(K_5)\leq4\).
    We know that \(\kappa(K_n)=n-1\) so we get that \(\kappa(K_5)=4>3\).
    This is a counterexample since \(K_5\) is 3-connected but it has vertex connectivity 4.
    The statement ``Every 3-connected graph has connectivity 3.'' is false. \\
    (c) The statement ``Every \(k\)-connected graph is \(k\)-edge-connected.'' is true. \\
    (d) The statement ``Every \(k\)-edge-connected graph is \(k\)-connected.'' is false.
\end{solution}

\subsection*{Chapter 5 Coloring of Graphs}
\subsection*{Section 5.1 Vertex Coloring and Upper Bounds}
\subsection*{Section 5.2 Structure of \(k\)-chromatic Graphs}
\subsection*{Section 5.3 Enumerative Aspects}
\end{document}