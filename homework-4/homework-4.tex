\documentclass[12pt]{article}
\usepackage[letterpaper, margin=1in]{geometry}
\usepackage{mlmodern}
\usepackage{amsmath}
\usepackage{amsthm}
\usepackage{amssymb}

\theoremstyle{definition}
\newtheorem{problem}{Problem}
\newenvironment*{solution}{\begin{proof}[Solution]}{\end{proof}}
\renewcommand*{\qedsymbol}{\(\blacksquare\)}

\title{MTH 4140 Graph Theory}
\author{Yaohui Wu}
\date{May 13, 2024}

\begin{document}
\maketitle
\section*{Homework 4}
\section*{Introduction to Graph Theory}

\subsection*{Chapter 4 Connectivity and Paths}
\subsection*{Section 4.1 Cuts and Connectivity}
\begin{problem}
    Excercise 4.1.1
\end{problem}
\begin{solution}
    (a) A graph is \(k\)-connected if its connectivity is at least \(k\).
    If a graph \(G\) is 2-connected, then its connectivity is at least 2.
    Given \(G\) has connectivity 4 so we have \(\kappa(G)=4\geq2\) which is
    true.
    The statement ``Every graph with connectivity 4 is 2-connected." is
    true. \\
    (b) A 3-connected graph \(G\) has connectivity at least 3 so
    \(\kappa(G)\geq3\).
    Consider the graph \(K_5\), we know that \(\kappa(G)\leq\delta(G)\) so
    \(\kappa(K_5)\leq4\).
    We know that \(\kappa(K_n)=n-1\) so we get that \(\kappa(K_5)=4>3\).
    This is a counterexample since \(K_5\) is 3-connected but it has
    connectivity 4.
    The statement ``Every 3-connected graph has connectivity 3.'' is false. \\
    (c) A graph is \(k\)-edge-connected if every disconnecting set has at
    least \(k\) edges.
    Consider a \(k\)-connected graph \(G\), we know that
    \(\kappa(G)\leq\kappa'(G)\) from Whitney's theorem.
    We can deduce that \[k \leq \kappa(G) \leq \kappa'(G) \]
    The edge-connectivity \(\kappa'(G)\) is the minimum size of a
    disconnecting set and it is at least \(k\) so \(G\) is
    \(k\)-edge-connected.
    The statement ``Every \(k\)-connected graph is \(k\)-edge-connected.'' is
    true. \\
    (d) Consider a \(k\)-edge-connected graph \(G\) with connectivity
    \(\kappa(G)\) and \\ edge-connectivity \(\kappa'(G)\).
    We have \(\kappa(G)\leq\kappa'(G)\) and \(k\leq\kappa'(G)\).
    If \(k>\kappa(G)\), then we have \[\kappa(G)<k\leq\kappa'(G)\] and
    Whitney's theorem still holds so \(G\) is not \(k\)-connected. \\
    The statement ``Every \(k\)-edge-connected graph is \(k\)-connected.'' is
    false.
\end{solution}

\begin{problem}
    Excercise 4.1.8(b)
\end{problem}
\begin{solution}
    Let \(G\) be the graph on the right, we see that \(G\) is a 4-regular
    graph so the minimum degree of \(G\) is 4.
    Since every vertex is connected to four other vertices,
    we must remove 4 incidental edges of any vertex at minimum to disconnect
    the graph.
    Similarly, we must remove 4 adjacent vertices of any vertex at minimum to
    disconnect the graph.
    Therefore, the edge-connectivity of \(G\) is 4 and the connectivity of
    \(G\) is 4.
    The connectivity \(\kappa(G)\), edge-connectivity \(\kappa'(G)\), and the
    minimum degree \(\delta(G)\) are
    \[\kappa(G) = \kappa'(G) = \delta(G) = 4\] which follows immediately from
    Whitney's theorem.
\end{solution}

\begin{problem}
    Excercise 4.1.11
\end{problem}
\begin{solution}
    Let \(S\) be a minimum vertex cut and since \(\kappa(G)\leq\kappa'(G)\) by
    Whitney's theorem, we only need an edge cut of size \(|S|=\kappa(G)\).
    Let \(H_1\) and \(H_2\) be two components of \(G-S\).
    Since \(S\) is minimum, every vertex \(v\) in \(S\) has a neighbor in \(H_1\)
    and a neighbor in \(H_2\).
    Given \(\Delta(G)\leq3\), \(v\) cannot have two neighbors in \(H_1\) and two
    neighbors in \(H_2\).
    For each \(v\in S\), delete the incidental edge in the component where
    \(v\) has only one neighbor.
    If all incidnetal edges to both components satisfy the condition, then we
    simply delete all incidental edges to one of the components.
    Thus, we get a disconnecting set of edges \(S'\) with minimum size
    \(|S'|=\kappa'(G)=|S|=\kappa(G)\).
    It is proved that \(\kappa'(G)=\kappa(G)\) when \(G\) is a simple graph
    with \(\Delta(G)\leq3\).
\end{solution}

\subsection*{Chapter 5 Coloring of Graphs}
\subsection*{Section 5.1 Vertex Coloring and Upper Bounds}
\begin{problem}
    Excercise 5.1.14
\end{problem}
\begin{solution}
    We can use the same color for vertices in the maximum independent set.
    Then we can use different colors for rest of the vertices.
    Therefore, we get a proper coloring of the graph.
    It is proved that for every graph \(G\), \(\chi(G) \leq n(G)-\alpha(G)+1\)
    is true.
\end{solution}

\begin{problem}
    Excercise 5.1.20
\end{problem}
\begin{solution}
    
\end{solution}

\begin{problem}
    5.1.41
\end{problem}
\begin{solution}
    
\end{solution}

\subsection*{Section 5.2 Structure of \(k\)-chromatic Graphs}
\begin{problem}
    Excercise 5.2.1
\end{problem}
\begin{solution}
    Suppose \(G\) is not a complete graph such that
    \(\chi(G-x-y) = \chi(G)-2\).
    Let \(x,y\) be two non-adjacent vertices so we can color them with the
    same color.
    It follows that \[\chi(G) = \chi(G-x-y)+1\] which implies that
    \[\chi(G-x-y) = \chi(G)-1\]
    Hence, there is a contradiction thus every pair of distinct vertices in
    \(G\) must be adjacent which implies that \(G\) is a complete graph.
    Therefore, if \(\chi(G-x-y) = \chi(G)-2\), then \(G\) is a complete graph.
\end{solution}

\begin{problem}
    Excercise 5.2.22
\end{problem}
\begin{solution}
    
\end{solution}

\subsection*{Section 5.3 Enumerative Aspects}
\begin{problem}
    Excercise 5.3.3
\end{problem}
\begin{solution}
    It is obvious that if \(k=2\), then we have
    \[k^4-4k^3+3k^2 = (2)^4-4(2)^3+3(2)^2 = 16-32+12 = -4\]
    The polynomial cannot count the proper 2-colorings of any graph.
    Therefore, it is proved that \(k^4-4k^3+3k^2\) is not a chromatic
    polynomial.
\end{solution}

\begin{problem}
    Excercise 5.3.4(a)
\end{problem}
\begin{solution}
    We can use mathematical induction to prove that the formula is true.
    The least vertices a simple cycle has is 3.
    If \(n=3\), then \(\chi(C_3;k)=k(k-1)(k-2)=k^3-3k^2+2k\) by counting the
    number of ways to choose the proper color of each vertex.
    The polynomial gives \((k-1)^3+(-1)^3(k-1)=k^3-3k^2+2k\) when \(n=3\) so
    it is true.
    For every \(n\geq3\), assume that \(\chi(C_n;k) = (k-1)^n+(-1)^n(k-1)\) is
    true.
    If we remove an edge from a cycle of \(n\) vertices, then we get a path of
    \(n\) vertices.
    If we contract an edge from a cycle of \(n\) vertices, then we get a cycle
    of \(n-1\) vertices.
    The chromatic recurrence gives \(\chi(G;k)=\chi(G-e;k)-\chi(G\cdot e;k)\)
    and the chromatic polynomial of a tree \(T\) with \(n\) vertices is
    \(\chi(T;k)=k(k-1)^{n-1}\).
    The chromatic polynomial of \(C_{n+1}\) is
    \begin{align*}
        \chi(C_{n+1};k) &= \chi(C_{n+1}-e;k) - \chi(C_{n+1}\cdot e;k)
        = \chi(P_{n+1};k) - \chi(C_{n};k) \\
        &= k(k-1)^{(n+1)-1} - [(k-1)^n+(-1)^n(k-1)] \\
        &= k(k-1)^n - (k-1)^n + (-1)^{n+1}(k-1) \\
        &= (k-1)(k-1)^n + (-1)^{n+1}(k-1) \\
        &= (k-1)^{n+1} + (-1)^{n+1}(k-1)
    \end{align*}
    which implies that the formula is true.
    It is proved that \(\chi(C_n;k) = (k-1)^n+(-1)^n(k-1)\).
\end{solution}

\begin{problem}
    Excercise 5.3.5
\end{problem}
\begin{solution}
    If \(n=1\), then we have a path of 2 vertices so \(\chi(G_1;k)=k(k-1)\)
    and the polynomial gives
    \[(k^2-3k+3)^{n-1}k(k-1) = (k^2-3k+3)^{1-1}k(k-1) = k(k-1)\]
    which is true.
    Assume that the polynomial is true for \(n\geq1\).
    Observe that \(\chi(G_n;k)=(k^2-3k+3)\chi(G_{n-1};k)\) by counting.
    Therefore, we have
    \begin{align*}
        \chi(G_{n+1};k) &= (k^2-3k+3)\chi(G_{n};k)
        = (k^2-3k+3)(k^2-3k+3)^{n-1}k(k-1) \\
        &= (k^2-3k+3)^nk(k-1)
    \end{align*}
    It is proved that \(\chi(G_n;k) = (k^2-3k+3)^{n-1}k(k-1)\).
\end{solution}

\begin{problem}
    Excercise 5.3.11
\end{problem}
\begin{solution}
    The sum of the coefficients of a chromatic polynomial implies that \(k=1\)
    so the sum is the number of proper colorings with exactly one color which
    must be 0 if the graph has edges.
    It is proved that the sum of the coefficients of \(\chi(G;k)\) is 0 unless
    \(G\) has no edges.
\end{solution}
\end{document}