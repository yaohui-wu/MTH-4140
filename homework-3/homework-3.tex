\documentclass[12pt]{article}
\usepackage[letterpaper, margin=1in]{geometry}
\usepackage{mlmodern}
\usepackage{amsmath}
\usepackage{amsthm}
\usepackage{amssymb}

\theoremstyle{definition}
\newtheorem{problem}{Problem}
\newenvironment*{solution}{\begin{proof}[Solution]}{\end{proof}}
\renewcommand*{\qedsymbol}{\(\blacksquare\)}

\title{MTH 4140 Graph Theory}
\author{Yaohui Wu}
\date{April 30, 2024}

\begin{document}
\maketitle
\section*{Homework 3}
\section*{Introduction to Graph Theory}

\subsection*{Chapter 2 Trees and Distances}
\subsubsection*{Section 2.2 Spanning Trees and Enumeration}
\begin{problem}
    Exercise 2.2.1
\end{problem}
\begin{solution}
    (a) The trees with Prüfer codes that has only one value are the stars. (b)
    The trees with Prüfer codes that has exactly two values are the double
    stars. (c) The trees with Prüfer codes that have distinct values in all
    positions are the paths.
\end{solution}
\begin{problem}
    Exercise 2.2.8(a)
\end{problem}
\begin{solution}
    We can choose two leaves in \(\binom{n}{2}\) ways since they do not appear
    in the Prüfer code. There are \((n-2)!\) permutations of the labels in the
    Prüfer code. There are
    \[\binom{n}{2}(n-2)! = \frac{n(n-1)(n-2)!}{2} = \frac{n!}{2}\]
    trees with only two leaves. Trees that have only two leaves are
    the paths. There are \(n!\) permutations of the vertices in a path with
    vertex set \([n]\). We adjust for overcount of isomorphic paths so there
    are \(\dfrac{n!}{2}\) paths. The number of trees with vertex set \([n]\)
    that have only two leaves is \(\dfrac{n!}{2}\).
\end{solution}

\subsubsection*{Section 2.3 Optimization and Trees}
\begin{problem}
    Exercise 2.3.3
\end{problem}
\begin{solution}
    We run Kruskal's algorithm to get a minimum spanning tree (MST) of the
    graph built from the given adjacency matrix. The total weight of the MST
    is \[3+3+7+8=21\] The least cost of making all the cities reachable from
    each other is 21.
\end{solution}
\begin{problem}
    Exercise 2.3.7
\end{problem}
\begin{solution}
    Suppose a graph \(G\) with distinct edge weights has two different minimum
    spanning trees \(T\) and \(T'\). Let \(e\) be the lightest edge of the
    symmetric difference. Since the edge weights are distinct, so \(e\) is in
    exactly one of the two trees. WLOG, if \(e\) is in \(E(T)\), then there
    exists an edge \(e'\) in \(E(T')\) in the symmetric difference such that
    \(T'-e'+e\) is a spanning tree. Thus, we have \(w(e)<w(e')\) so
    \[w(T'-e'+e) < w(T')\] which is a contradiction since we assumed that
    \(T'\) is an MST. It is proved that a weighted connected graph \(G\) with
    distinct edge weights has only one minimum spanning tree.
\end{solution}
\begin{problem}
    Exercise 2.3.26
\end{problem}
\begin{solution}
    Let \(a_n\) be the recurrence relation of the number of binary trees with
    \(n+1\) leaves. There is only one binary tree with one leaf when \(n=0\)
    so \(a_0=1\). In general when \(n>0\), for each tree there are \(k\)
    leaves in the left subtree and \(n-k+1\) leaves in the right subtree from
    the root where \(1\leq k\leq n\). The recurrence relation is
    \begin{align*}
        a_0 &= 1 \\ a_n &= \sum_{k=1}^n a_{k-1}a_{n-k},n>0
    \end{align*}
\end{solution}

\subsection*{Chapter 3 Matchings and Factors}
\subsubsection*{Section 3.1 Matchings and Covers}
\begin{problem}
    Exercise 3.1.2
\end{problem}
\begin{solution}
    Suppose that \(C_n\) has a maximal matching \(M\) of size \(k\). There are
    \(k\) matched edges that map to \(n-2k\) unsaturated vertices in \(M\).
    Since the matching is maximal, every matched edge maps to at most one
    unsaturated vertex which implies that this is a set of at most 3 vertices.
    By the Pigeonhole principle, there are
    \(\left\lceil \dfrac{n}{3} \right\rceil\) such sets in a cycle \(C_n\) of
    \(n\) vertices. The minimum size of a maximal matching in the cycle
    \(C_n\) is \(\left\lceil \dfrac{n}{3} \right\rceil\).
\end{solution}
\begin{problem}
    Exercise 3.1.8
\end{problem}
\begin{solution}
    Suppose that a tree \(T\) has two different perfect matchings \(M\) and
    \(M'\). Consider the symmetric difference of the edge sets of \(M\) and
    \(M'\) denoted by \(M\triangle M'\). If a vertex is incidental to the same
    edge in both perfect matchings, then it is an isolated vertex in the
    symmetric difference. Otherwise, the vertex is incidental to two different
    edges in the symmetric difference. Hence, every vertex of
    \(M\triangle M'\) must has degree 0 or 2. We know that every component of
    \(M\triangle M'\) must be a path or an even cycle. Since \(T\) is a tree
    and trees have no cycles so every component of \(M\triangle M'\) must be
    an isolated vertex with degree 0. This implies that the two perfect
    matchings are the same but there is a contradiction. Therefore, it is
    proved that every tree has at most one perfect matching.
\end{solution}
\begin{problem}
    Exercise 3.1.16
\end{problem}
\end{document}