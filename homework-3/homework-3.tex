\documentclass[12pt]{article}
\usepackage[letterpaper, margin=1in]{geometry}
\usepackage{mlmodern}
\usepackage{amsthm}
\usepackage{amssymb}

\theoremstyle{definition}
\newtheorem{problem}{Problem}
\newenvironment*{solution}{\begin{proof}[Solution]}{\end{proof}}
\renewcommand*{\qedsymbol}{\(\blacksquare\)}

\title{MTH 4140 Graph Theory}
\author{Yaohui Wu}
\date{April 24, 2024}

\begin{document}
\maketitle
\section*{Homework 3}
\section*{Introduction to Graph Theory}

\subsection*{Chapter 2 Trees and Distances}
\subsubsection*{Section 2.2 Spanning Trees and Enumeration}
\begin{problem}
    Exercise 2.2.1
\end{problem}
\begin{solution}
    (a) The trees with Prüfer codes that has only one value are the stars. (b)
    The trees with Prüfer codes that has exactly two values are the double
    stars. (c) The trees with Prüfer codes that have distinct values in all
    positions are the paths.
\end{solution}
\begin{problem}
    Exercise 2.2.8(a)
\end{problem}

\subsubsection*{Section 2.3 Optimization and Trees}
\begin{problem}
    Exercise 2.3.3
\end{problem}
\begin{solution}
    We run Kruskal's algorithm to get a minimum spanning tree (MST) of the
    graph built from the given adjacency matrix. The total weight of the MST
    is \[3+3+7+8=21\] The least cost of making all the cities reachable from
    each other is 21.
\end{solution}
\begin{problem}
    Exercise 2.3.7
\end{problem}
\begin{problem}
    Exercise 2.3.26
\end{problem}

\subsection*{Chapter 3 Matchings and Factors}
\subsubsection*{Section 3.1 Matchings and Covers}
\begin{problem}
    Exercise 3.1.2
\end{problem}
\begin{problem}
    Exercise 3.1.8
\end{problem}
\begin{solution}
    Suppose that a tree \(T\) has two different perfect matchings \(M\) and
    \(M'\). Consider the symmetric difference of the edge sets of \(M\) and
    \(M'\) denoted by \(M\triangle M'\). If a vertex is incidental to the same
    edge in both perfect matchings, then it is an isolated vertex in the
    symmetric difference. Otherwise, the vertex is incidental to two different
    edges in the symmetric difference. Hence, every vertex of
    \(M\triangle M'\) must has degree 0 or 2. We know that every component of
    \(M\triangle M'\) must be a path or an even cycle. Since \(T\) is a tree
    and trees have no cycles so every component of \(M\triangle M'\) must be
    an isolated vertex with degree 0. This implies that the two perfect
    matchings are the same but there is a contradiction. Therefore, it is
    proved that every tree has at most one perfect matching.
\end{solution}
\begin{problem}
    Exercise 3.1.16
\end{problem}
\end{document}