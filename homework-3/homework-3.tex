\documentclass[12pt]{article}
\usepackage[letterpaper, margin=1in]{geometry}
\usepackage{mlmodern}
\usepackage{amsthm}
\usepackage{amssymb}

\theoremstyle{definition}
\newtheorem{problem}{Problem}
\newenvironment*{solution}{\begin{proof}[Solution]}{\end{proof}}
\renewcommand*{\qedsymbol}{\(\blacksquare\)}

\title{MTH 4140 Graph Theory}
\author{Yaohui Wu}
\date{April 24, 2024}

\begin{document}
\maketitle
\section*{Homework 3}
\section*{Introduction to Graph Theory}

\subsection*{Chapter 2 Trees and Distances}
\subsubsection*{Section 2.2 Spanning Trees and Enumeration}
\begin{problem}
    Exercise 2.2.1
\end{problem}
\begin{solution}
    (a) The trees with Prüfer codes that has only one value are the stars. (b)
    The trees with Prüfer codes that has exactly two values are the double
    stars. (c) The trees with Prüfer codes that have distinct values in all
    positions are the paths.
\end{solution}

\subsubsection*{Section 2.3 Optimization and Trees}
\begin{problem}
    Exercise 2.3.8
\end{problem}
\begin{solution}
    We run Kruskal's algorithm to get a minimum spanning tree (MST) of the
    graph built from the given adjacency matrix. The total weight of the MST
    is \[3+3+7+8=21\] The least cost of making all the cities reachable from
    each other is 21.
\end{solution}

\subsection*{Chapter 3 Matchings and Factors}
\subsubsection*{Section 3.1 Matchings and Covers}
\end{document}