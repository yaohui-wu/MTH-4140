\documentclass[12pt, letterpaper]{article}
\usepackage[margin=1in]{geometry}
\usepackage{amsmath}
\usepackage{amsthm}
\usepackage{amssymb}

\newenvironment{solution}{\begin{proof}[Solution]}{\end{proof}}
\renewcommand*{\qedsymbol}{\(\blacksquare\)}

\title{MTH 4140 Homework 1}
\author{Yaohui Wu}
\date{February 14, 2024}

\begin{document}
\maketitle
\section*{Problem 1}
\subsection*{Section 1.1 Problem 9}
\begin{solution}
    Let the graph on the left be \(G\) and the graph on the right be \(H\). It
    is equivalent to show that \(G\cong\overline{H}\). If we label every vertex
    in \(G\) and \(H\) using the same set of labels then two vertices are
    adjacent in \(G\) if and only if they are adjacent in \(\overline{H}\).
    Therefore, it is proved that \(G\cong\overline{H}\) so \(\overline{G}\cong H\).
\end{solution}

\section*{Problem 2}
\subsection*{Section 1.1 Problem 10}
\begin{solution}
    We have a disconnected graph \(G\) and its complement \(\overline{G}\). Let
    \(u, v\) be two vertices of different components in \(G\) s.t. there is no
    path from \(u\) to \(v\). Then \(u, v\) are adjacent in \(\overline{G}\) and
    all other vertices in \(\overline{G}\) are adjacent to one of \(u, v\) or
    both since \(u, v\) have no common neighbors. There is a path for any two
    vertices in \(\overline{G}\) using \(u,v\). Therefore, it is proved that the
    complement of a simple disconnected graph must be connected.
\end{solution}

\section*{Problem 3}
\subsection*{Section 1.1 Problem 17}
\begin{solution}
    We know that \(G\cong H\iff\overline{G}\cong\overline{H}\) so it is
    equivalent to count all 2-regular graphs on 7 vertices. Only the graph
    \(C_7\) and the graph with only \(C_3,C_4\) satisfy this. Therefore, the
    number of isomorphic classes of simple 7-vertex 4-regular graphs is 2.
\end{solution}

\section*{Problem 4}
\subsection*{Section 1.1 Problem 25}
\begin{solution}
    Suppose the Peterson graph has \(C_7\) then there are 3 remaining vertices
    outside of \(C_7\). The Peterson graph is 3-regular so each of the 7
    vertices on \(C_7\) has one extra edge outside of \(C_7\). The Peterson
    graph has girth 5 so the 7 extra edges have to have one of the 3 remaining
    vertices as an endpoint. By the pigeonhole principle, at least 3 of the
    extra edges end on the same vertex. WLOG, we will have a \(C_3\) but the
    Peterson graph has girth 5 so there is a contradiction. Therefore, it is
    proved that the Peterson graph does not have a cycle of length 7.
\end{solution}

\section*{Problem 5}
\subsection*{Section 1.1 Problem 26}
\begin{solution}
    If \(G\) has girth 4 then there are two adjacent vertices \(u, v\) with no
    common neighbors. Since \(G\) is k-regular and \(u,v\) each has degree
    \(k-1\), the sum of the degrees is at least \(k-1+1+1+k-1=2k\). Therefore,
    it is proved that \(G\) has at least \(2k\) vertices. If each vertex in the
    independent set \(\{x_1,\dots,x_{k}\}\) is adjacent to all vertices of
    the independent set \(\{y_1,\dots,y_{k}\}\) then we have exactly \(2k\)
    vertices. All such graphs with excatly \(2k\) vertices must be complete
    bipartite graphs \(K_{n,n}\) where \(n=k\).
\end{solution}
\end{document}