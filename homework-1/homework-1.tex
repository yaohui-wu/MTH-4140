\documentclass[12pt]{article}
\usepackage[letterpaper, margin=1in]{geometry}
\usepackage{amsmath}
\usepackage{amsthm}
\usepackage{amssymb}

\newenvironment{solution}{\begin{proof}[Solution]}{\end{proof}}
\renewcommand*{\qedsymbol}{\(\blacksquare\)}

\title{MTH 4140 Homework 1}
\author{Yaohui Wu}
\date{February 18, 2024}

\begin{document}
\maketitle
\section*{Problem 1}
\subsection*{Section 1.1 Problem 8}
\begin{solution}
    Let the graph on the left be \(G\) and we label its vertices using
    \(\{a,b,c,d,e,f,g,h\}\). The vertices for the outer square are \(\{a,b\}\)
    from left to right on the top and \(\{c,d\}\) from left to right on the
    bottom respectively. Similarly, the vertices for the inner square are
    \(\{e,f\}\) from left to right on the top and \(\{g,h\}\) from left to right
    on the bottom respectively. We can decompose \(G\) into 4 copies of \(K_3\)
    with the center being the vertex of degree 3. They are \(\{a,e,f,g\}\) with
    center \(e\), \(\{a,b,d,f\}\) with center \(b\), \(\{d,f,g,h\}\) with center
    \(h\), and \(\{a,c,d,g\}\) with center \(c\). We can also decompose \(G\)
    into 4 copies of \(P_4\). They are the \(a,e\)-path \(\{a,b,f,e\}\), the
    \(b,f\)-path \(\{b,d,h,f\}\), the \(d,h\)-path \(\{d,c,g,h\}\), and the
    \(c,g\)-path \(\{c,a,e,g\}\). Therefore, it is shown that \(G\) decomposes
    into copies of \(K_3\) and also into copies of \(P_4\).
\end{solution}

\section*{Problem 2}
\subsection*{Section 1.1 Problem 9}
\begin{solution}
    Let the graph on the left be \(G\) and the graph on the right be \(H\). It
    is equivalent to show that \(G\cong\overline{H}\). If we label every vertex
    in \(G\) and \(H\) using the same set of labels then two vertices are
    adjacent in \(G\) if and only if they are adjacent in \(\overline{H}\).
    Therefore, we have \(G\cong\overline{H}\) so it is proved that \(\overline{G}\cong H\).
\end{solution}

\section*{Problem 3}
\subsection*{Section 1.1 Problem 10}
\begin{solution}
    We have a disconnected graph \(G\) and its complement \(\overline{G}\). Let
    \(u, v\) be two vertices of different components in \(G\) s.t. there is no
    path from \(u\) to \(v\). Then \(u, v\) are adjacent in \(\overline{G}\) and
    all other vertices in \(\overline{G}\) are adjacent to one of \(u, v\) or
    both since \(u, v\) have no common neighbors. There is a path for any two
    vertices in \(\overline{G}\) using \(u,v\). Therefore, it is proved that the
    complement of a simple disconnected graph must be connected.
\end{solution}

\section*{Problem 4}
\subsection*{Section 1.1 Problem 11}
\begin{solution}
    Let's label the vertices in the middle as \(\{a,b,c,d\}\) from left to right respectively,
    the top vertex as \(e\), and the bottom vertex as \(f\). We can see that the
    graph \(G\) has a clique of size 4 with vertices \(\{a,b,e,f\}\). Suppose
    \(G\) has a clique of size greater than 4 then there must be at least 5 vertices
    with a degree of at least 4. However, \(G\) only has 4 vertices with a degree
    of at least 4 so there is a contradiction. Therefore, it is shown that the
    maximum size of a clique in \(G\) is 4. All of the independent sets in \(G\)
    are \(\{a,c\},\{a,d\},\{b,d\},\{e\},\{f\}\). There is not a set of three or more
    vertices s.t. any two vertices in the set are not adjacent. Therefore, it
    is shown that the maximum size of an independent set in \(G\) is 2.
\end{solution}

\section*{Problem 5}
\subsection*{Section 1.1 Problem 12}
\begin{solution}
    The Petersen graph is not bipartite because it has odd cycles. The independent
    set of the Petersen graph must have vertices that are 2-element subsets of
    \(\{1,2,3,4,5\}\) with a common element. The 2-element subsets of the largest
    independent set are the combination of any element \(n\in\{1,2,3,4,5\}\) and
    the rest of the 4 elements in \(\{1,2,3,4,5\}-\{n\}\). Therefore, the size of
    the largest independent set of the Petersen graph is 4.
\end{solution}

\section*{Problem 6}
\subsection*{Section 1.1 Problem 17}
\begin{solution}
    We know that \(G\cong H\iff\overline{G}\cong\overline{H}\) so it is
    equivalent to count all 2-regular graphs on 7 vertices. Only the graph
    \(C_7\) and the graph with only \(C_3,C_4\) satisfy this. Therefore, the
    number of isomorphic classes of simple 7-vertex 4-regular graphs is 2.
\end{solution}

\section*{Problem 7}
\subsection*{Section 1.1 Problem 19}
\begin{solution}
    Let the graphs be \(G,H,F\) from left to right respectively. There is a
    corresponding relationship between the vertices of \(H\) and \(F\). If we
    label all vertices in one of the two graphs then we can also label all of
    the vertices in the other graph to have the same properties of adjacency.
    However, this is not true for \(G\) because \(G\) is bipartite and \(H,F\)
    are not bipartite. Therefore, it is shown that \(H\cong F\).
\end{solution}

\section*{Problem 8}
\subsection*{Section 1.1 Problem 25}
\begin{solution}
    Suppose the Petersen graph has \(C_7\) then there are 3 remaining vertices
    outside of \(C_7\). The Petersen graph is 3-regular so each of the 7
    vertices on \(C_7\) has one extra edge outside of \(C_7\). The Petersen
    graph has girth 5 so the 7 extra edges have to have one of the 3 remaining
    vertices as an endpoint. By the pigeonhole principle, at least 3 of the
    extra edges end on the same vertex. WLOG, we will have a \(C_3\) but the
    Petersen graph has girth 5 so there is a contradiction. Therefore, it is
    proved that the Petersen graph does not have a cycle of length 7.
\end{solution}

\section*{Problem 9}
\subsection*{Section 1.1 Problem 26}
\begin{solution}
    If \(G\) has girth 4 then there are two adjacent vertices \(u, v\) with no
    common neighbors. Since \(G\) is k-regular and \(u,v\) each has degree
    \(k-1\), the sum of the degrees is at least \(k-1+1+1+k-1=2k\). Therefore,
    it is proved that \(G\) has at least \(2k\) vertices. If each vertex in the
    independent set \(\{x_1,\dots,x_{k}\}\) is adjacent to all vertices of
    the independent set \(\{y_1,\dots,y_{k}\}\) then we have exactly \(2k\)
    vertices. All such graphs with exactly \(2k\) vertices must be complete
    bipartite graphs \(K_{n,n}\) where \(n=k\).
\end{solution}

\section{Problem 10}
\subsubsection*{Section 1.2 Problem 3}
\begin{solution}
    The graph \(G\) has 4 components and they are \(\{1\},\{11\},\{13\}\), and \\
    \(\{2,3,4,5,6,7,8,9,10,12,14,15\}\). We have a \(7,5\)-path \(\{7,14,2,4,8,10,12,3,6,9,15,5\}\)
    because any two consecutive integers both have a greatest common factor
    greater than 1. This path travels every vertex in the largest component of
    \(G\) so it is the path with the longest length. The maximum length of a
    path in \(G\) is 12.
\end{solution}

\section*{Problem 11}
\subsection*{Problem A}
\begin{solution}
    Let the three husbands be \(A,B,C\) and their wives be \(a,b,c\) respectively.
    The steps to cross the river from \(L\) to \(R\) are
    \begin{enumerate}
        \item \(A,a\rightarrow R\)
        \item \(L\leftarrow A\)
        \item \(b,c\rightarrow R\)
        \item \(L\leftarrow a\)
        \item \(B,C\rightarrow R\)
        \item \(L\leftarrow B,b\)
        \item \(A,B\rightarrow R\)
        \item \(L\leftarrow c\)
        \item \(a,c\rightarrow R\)
        \item \(L\leftarrow B\)
        \item \(B,b\rightarrow R\)
    \end{enumerate}
\end{solution}

\section*{Problem 12}
\subsection*{Problem B}
\begin{solution}
    Let \(K_n\) be a simple connected complete graph of \(n\) vertices s.t.
    every vertex has degree \(n-1\) and \(K_n\) has \(\binom{n}{2}=\frac{n(n-1)}{2}\)
    edges. \(K_n\) has the maximum number of edges that a simple connected graph
    of \(n\) vertices can have because any two vertices in \(K_n\) are adjacent
    hence there is an edge for any pair of vertices. If \(G\) has the maximum
    number of edges then it must have two components, \(K_9\) and an isolated
    vertex \(v\). The number of edges of \(G\) is \(\frac{9(9-1)}{2}=36\).
    Therefore, it is proved that \(G\) has at most 36 edges.
\end{solution}
\end{document}