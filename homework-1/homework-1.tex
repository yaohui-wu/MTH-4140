\documentclass[12pt, letterpaper]{article}
\usepackage[margin=1in]{geometry}
\usepackage{amsmath}
\usepackage{amsthm}
\usepackage{amssymb}

\newenvironment{solution}{\begin{proof}[Solution]}{\end{proof}}
\renewcommand*{\qedsymbol}{\(\blacksquare\)}

\title{MTH 4140 Homework 1}
\author{Yaohui Wu}
\date{February 14, 2024}

\begin{document}
\maketitle
\section*{Problem 1}
\subsection*{Section 1.1 Problem 9}
\begin{solution}
    Let the graph on the left be \(G\) and the graph on the right be \(H\). It
    is equivalent to show that \(G\cong\overline{H}\). If we label every vertex
    in \(G\) then two vertices are adjacent if and only if they are adjacent in
    \(\overline{H}\). Therefore, it is proved that \(G\cong\overline{H}\) so
    \(\overline{G}\cong H\).
\end{solution}

\section*{Problem 2}
\subsection*{Section 1.1 Problem 17}
\begin{solution}
    We know that \(G\cong H\iff\overline{G}\cong\overline{H}\) so it is
    equivalent to count all 2-regular graphs on 7 vertices. Only the graph
    \(C_7\) and the graph with only \(C_3,C_4\) satisfy this. Therefore, the
    number of isomorphic classes of simple 7-vertex 4-regular graphs is 2.
\end{solution}

\section*{Problem 3}
\subsection*{Section 1.1 Problem 25}
\begin{solution}
    Suppose the Peterson graph has \(C_7\) then there are 3 remaining vertices
    outside of \(C_7\). The Peterson graph is 3-regular so each of the 7
    vertices on \(C_7\) has one extra edge outside of \(C_7\). The Peterson
    graph has girth 5 so the 7 extra edges have to have one of the 3 remaining
    vertices as an endpoint. By the pigeonhole principle, at least 3 of the
    extra edges end on the same vertex. WLOG, we will have a \(C_3\) but the
    Peterson graph has girth 5 so there is a contradiction. Therefore, it is
    proved that the Peterson graph does not have a cycle of length 7.
\end{solution}
\end{document}