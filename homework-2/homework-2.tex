\documentclass[12pt]{article}
\usepackage[letterpaper, margin=1in]{geometry}
\usepackage{amsmath}
\usepackage{amsthm}
\usepackage{amssymb}
\usepackage{enumerate}

\newenvironment*{solution}{\begin{proof}[Solution]}{\end{proof}}
\renewcommand*{\qedsymbol}{\(\blacksquare\)}

\title{MTH 4140 Homework 2}
\author{Yaohui Wu}
\date{March 10, 2024}

\begin{document}
\maketitle
\section*{Problem 1}
\subsection*{Section 1.3 Problem 8}
\begin{solution}
    We apply the Havel-Hakimi algorithm then we have
    \begin{enumerate}[(a)]
        \item \((5,5,4,3,2,2,2,1)\implies(4,3,2,1,1,2,1)\implies(4,3,2,2,1,1,1
        )\implies(2,1,1,0,1,1) \\ \implies(2,1,1,1,1,0)\implies(0,0,1,1,0)
        \implies(1,1,0,0,0)\implies(0,0,0,0)\)
        \item \((5,5,4,4,2,2,1,1)\implies(4,3,3,1,1,1,1)\implies(2,2,0,0,1,1)
        \implies(2,2,1,1,0,0)\implies(1,0,1,0,0)\implies(1,1,0,0,0)\implies(0,
        0,0,0)\)
        \item \((5,5,5,3,2,2,1,1)\implies(4,4,2,1,1,1,1)\implies(3,1,0,0,1,1)
        \implies(3,1,1,1,0,0)\implies(0,0,0,0,0)\)
        \item \((5,5,5,4,2,1,1,1)\implies(4,4,3,1,0,1,1)\implies(4,4,3,1,1,1,0
        )\ \\ implies(3,2,0,0,1,0)\implies(3,2,1,0,0,0)\implies(1,0,-1,0,0)\)
    \end{enumerate}
    Therefore, it is shown that \((5,5,4,3,2,2,2,1),(5,5,4,4,2,2,1,1),(5,5,5,3
    ,2,2,1,1)\) \\ are graphic sequences but \((5,5,5,4,2,1,1,1)\)  is not a
    graphic sequence.
\end{solution}
\section*{Problem 2}
\subsection*{Section 1.3 Problem 9}
\begin{solution}
    Let \(G\) be a graph of 26 vertices from two sets of 13 vertices s.t. the
    vertices represent the teams, the sets represent the divisions, and there
    is an edge between two vertices if two teams play against each other. Let
    \(H\) be a subgraph of \(G\) with 13 vertices in the same set s.t. it
    represent 13 teams in the same division. Suppose the proposition is true,
    each of the 13 vertices in \(H\) will have degree \(9\). Then the sum of
    the degrees in \(H\) must be \(13\cdot9\) which must be odd. However, the
    handshaking lemma states that for a graph \(G=(V,E)\) we have \(\sum_{deg
    (v)\in V}=2|E|\), i.e. the sum of the degrees is twice the number of edges
    which must be even. There is a contradiction and the proposition is false,
    that which was to be demonstrated. Therefore, it is proved that for a
    league with two divisions of 13 teams each, it is impossible to schedule a
    season with each team playing nine games against teams within its division
    and four games against teams in the other division.
\end{solution}
\section*{Problem 3}
\subsection*{Section 1.3 Problem 18}
\begin{solution}
    Suppose the proposition is true for the sake of contradiction. Let \(G\)
    be a \(k\)-regular bipartite graph that has a cut edge \(e\) s.t. \(G-e\)
    is not connected. Let \(A\) and \(B\) be the two independent sets of
    vertices in \(G\). We remove \(e\) from \(G\) then \(G-e\) becomes a
    disjoint union of 2 bipartite graphs \(G_1\) and \(G_2\). The endpoints of
    \(e\) are the vertex \(u\in A\) in \(G_1\) and the vertex \(v\in B\) in
    \(G_2\). \(G_1\) has \(a\) vertices with \(a-1\) vertices having degree
    \(k\) and the vertex \(u\) having degree \(k-1\) in \(A\). \(G_1\) also
    has \(b\) vertices with degree \(k\) in \(B\). Then we have
    \begin{align*}
        E(G_1)=(a-1)k+(k-1) &= b\cdot k \\ a\cdot k-k+k-1 &= b\cdot k \\ a
        \cdot k-1 &= b\cdot k \\ (a-b)k &= 1
    \end{align*}
    Since we are given that \(k\geq2\) so we have \((a-b)k\neq1\) hence there
    is a contradicition and the proposition is false. Therefore, it is proved
    that for \(k\geq2\), a \(k\)-regular bipartite graph has no cut-edge.
\end{solution}
\section*{Problem 4}
\subsection*{Section 1.3 Problem 61}
\begin{solution}
    Let \((d_1,d_2,\dots,d_n)\) be the degree sequence of the graph \(G\) with
    \(n\) vertices. Then the degree sequence of \(\overline{G}\) is \((n-1-d_n
    ,\dots,n-1-d_2,n-1-d_1)\). Since \(G\cong\overline{G}\) then the two
    degree sequences of \(G\) and \(\overline{G}\) are exactly the same. Thus
    we have \[(d_1,d_2,\dots,d_n)=(n-1-d_n,\dots,n-1-d_2,n-1-d_1)\] Given
    that \(n\equiv1\pmod4\) hence we get that \(n\) is an odd number. Consider
    the middle elements in both degree sequences, they are equal thus we have
    \begin{align*}
        d_{\frac{n+1}{2}} &= n-1-d_{\frac{n+1}{2}} \\ 2\cdot d_{\frac{n+1}{2}}
        &= n-1 \\ d_{\frac{n+1}{2}} &= \frac{n-1}{2}
    \end{align*}
    Therefore, it is proved that if \(G\cong\overline{G}\) and the number of
    vertices in \(G\) is \(n\equiv1\pmod4\) then \(G\) has at least one vertex
    with degree \(\frac{n-1}{2}\).
\end{solution}
\end{document}