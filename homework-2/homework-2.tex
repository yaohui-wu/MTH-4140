\documentclass[12pt]{article}
\usepackage[letterpaper, margin=1in]{geometry}
\usepackage{amsthm}
\usepackage{amssymb}

\newenvironment*{solution}{\begin{proof}[solution]}{\end{proof}}
\renewcommand*{\qedsymbol}{\(\blacksquare\)}

\title{MTH 4140 Homework 2}
\author{Yaohui Wu}
\date{March 3, 2024}

\begin{document}
\maketitle
\section*{Problem 1}
\subsection*{Section 1.3 Problem 9}
\begin{solution}
    Let \(G\) be a graph of 26 vertices from two sets of 13 vertices s.t. the
    vertices represent the teams, the sets represent the divisions, and there
    is an edge between two vertices if two teams play against each other. Let
    \(H\) be a subgraph of \(G\) with 13 vertices in the same set s.t. it
    represent 13 teams in the same division. Suppose the proposition is true,
    each of the 13 vertices in \(H\) will have degree \(9\). Then the sum of
    the degrees in \(H\) must be \(13\cdot9\) which must be odd. However, the
    handshaking lemma states that for a graph \(G=(V,E)\) we have \(\sum_{deg
    (v)\in V}=2|E|\), i.e. the sum of the degrees is twice the number of edges
    which must be even. There is a contradiction and the proposition is false,
    that which was to be demonstrated. Therefore, it is proved that for a
    league with two divisions of 13 teams each, it is impossible to schedule a
    season with each team playing nine games against teams within its division
    and four games against teams in the other division.
\end{solution}
\end{document}