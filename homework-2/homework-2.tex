\documentclass[12pt]{article}
\usepackage[letterpaper, margin=1in]{geometry}
\usepackage{mlmodern}
\usepackage{amsmath}
\usepackage{amsthm}
\usepackage{amssymb}
\usepackage{enumerate}

\newenvironment*{solution}{\begin{proof}[Solution]}{\end{proof}}
\renewcommand*{\qedsymbol}{\(\blacksquare\)}

\title{MTH 4140 Homework 2}
\author{Yaohui Wu}
\date{March 28, 2024}

\begin{document}
\maketitle
\section*{Problem 1}
\subsection*{Section 1.3 Problem 5}
\begin{solution}
    We can choose a vertex \(v\) in \(2^k\) ways then we choose two of its
    neighbors in \(\binom{k}{2}\) ways to form a \(P_3\). There is four \(P_3
    \) for every \(C_4\) since we can choose an edge in a \(C_4\) to remove
    then we get a \(P_3\). Therefore, it is proved that the number of copies
    of \(P_3\) is \(\binom{k}{2}2^k\) and the number of copies of \(C_4\) is
    \(\frac{\binom{k}{2}2^k}{4}=\binom{k}{2}2^{k-2}\) in \(Q_k\).
\end{solution}
\section*{Problem 2}
\subsection*{Section 1.3 Problem 8}
\begin{solution}
    We apply the Havel-Hakimi algorithm then we have
    \begin{enumerate}[(a)]
        \item \((5,5,4,3,2,2,2,1)\implies(4,3,2,1,1,2,1)\implies(4,3,2,2,1,1,1
        )\implies(2,1,1,0,1,1) \\ \implies(2,1,1,1,1,0)\implies(0,0,1,1,0)
        \implies(1,1,0,0,0)\implies(0,0,0,0)\)
        \item \((5,5,4,4,2,2,1,1)\implies(4,3,3,1,1,1,1)\implies(2,2,0,0,1,1)
        \implies(2,2,1,1,0,0)\implies(1,0,1,0,0)\implies(1,1,0,0,0)\implies(0,
        0,0,0)\)
        \item \((5,5,5,3,2,2,1,1)\implies(4,4,2,1,1,1,1)\implies(3,1,0,0,1,1)
        \implies(3,1,1,1,0,0)\implies(0,0,0,0,0)\)
        \item \((5,5,5,4,2,1,1,1)\implies(4,4,3,1,0,1,1)\implies(4,4,3,1,1,1,0
        )\ \\ implies(3,2,0,0,1,0)\implies(3,2,1,0,0,0)\implies(1,0,-1,0,0)\)
    \end{enumerate}
    Therefore, it is proved that \((5,5,4,3,2,2,2,1),(5,5,4,4,2,2,1,1),(5,5,5,3
    ,2,2,1,1)\) \\ are graphic sequences but \((5,5,5,4,2,1,1,1)\)  is not a
    graphic sequence.
\end{solution}
\section*{Problem 3}
\subsection*{Section 1.3 Problem 9}
\begin{solution}
    Let \(G\) be a graph of 26 vertices from two sets of 13 vertices s.t. the
    vertices represent the teams, the sets represent the divisions, and there
    is an edge between two vertices if two teams play against each other. Let
    \(H\) be a subgraph of \(G\) with 13 vertices in the same set s.t. it
    represent 13 teams in the same division. Suppose the proposition is true,
    each of the 13 vertices in \(H\) will have degree \(9\). Then the sum of
    the degrees in \(H\) must be \(13\cdot9\) which must be odd. However, the
    handshaking lemma states that for a graph \(G=(V,E)\) we have \(\sum_{\deg
    (v)\in V}=2|E|\), i.e. the sum of the degrees is twice the number of edges
    which must be even. There is a contradiction and the proposition is false,
    that which was to be demonstrated. Therefore, it is proved that for a
    league with two divisions of 13 teams each, it is impossible to schedule a
    season with each team playing nine games against teams within its division
    and four games against teams in the other division.
\end{solution}
\section*{Problem 4}
\subsection*{Section 1.3 Problem 18}
\begin{solution}
    Suppose the proposition is true for the sake of contradiction. Let \(G\)
    be a \(k\)-regular bipartite graph that has a cut edge \(e\) s.t. \(G-e\)
    is not connected. Let \(A\) and \(B\) be the two independent sets of
    vertices in \(G\). We remove \(e\) from \(G\) then \(G-e\) becomes a
    disjoint union of 2 bipartite graphs \(G_1\) and \(G_2\). The endpoints of
    \(e\) are the vertex \(u\in A\) in \(G_1\) and the vertex \(v\in B\) in
    \(G_2\). \(G_1\) has \(a\) vertices with \(a-1\) vertices having degree
    \(k\) and the vertex \(u\) having degree \(k-1\) in \(A\). \(G_1\) also
    has \(b\) vertices with degree \(k\) in \(B\). Then we have
    \begin{align*}
        e(G_1)=(a-1)k+(k-1) &= b\cdot k \\ a\cdot k-k+k-1 &= b\cdot k \\ a
        \cdot k-1 &= b\cdot k \\ (a-b)k &= 1
    \end{align*}
    Since we are given that \(k\geq2\) so we have \((a-b)k\neq1\) hence there
    is a contradicition and the proposition is false. Therefore, it is proved
    that for \(k\geq2\), a \(k\)-regular bipartite graph has no cut-edge.
\end{solution}
\section*{Problem 5}
\subsection*{Section 1.3 Problem 45}
\begin{solution}
    The Kőnig's theorem states that a graph is bipartite if and only if it has
    no odd cycles. Let \(G\) be the Petersen graph, we proved that \(G\) has
    12 \(C_5\) and every edge of \(G\) belongs to 4 \(C_5\). Thus in order to
    remove 12 \(C_5\) from \(G\), we have to remove at least \(\frac{12}{4}=3
    \) edges. Since \(G\) has 15 edges hence we will have at most \(15-3=12\)
    edges in the bipartite subgraph of \(G\) after we removed all of the odd
    cycles. Therefore, it is proved that the maximum number of edges in a
    bipartite subgraph of the Petersen graph is 12.
\end{solution}
\section*{Problem 6}
\subsection*{Section 1.3 Problem 61}
\begin{solution}
    Let \((d_1,d_2,\dots,d_{n(G)})\) be the degree sequence of the graph \(G\)
    with \(n(G)\) vertices. Then the degree sequence of \(\overline{G}\) is \(
    (n-1-d_{n(G)},\dots,n(G)-1-d_2,n(G)-1-d_1)\). Since \(G\cong\overline{G}\)
    then the two degree sequences of \(G\) and \(\overline{G}\) are exactly
    the same. Thus we have \[(d_1,d_2,\dots,d_{n(G)})=(n(G)-1-d_{n(G)},\dots,
    n(G)-1-d_2,n(G)-1-d_1)\] Given that \(n(G)\equiv1\pmod4\) hence we have
    that \(n(G)\) is an odd number. Consider the middle elements in both
    degree sequences, they are equal thus we have
    \begin{align*}
        d_{\frac{n(G)+1}{2}} &= n(G)-1-d_{\frac{n(G)+1}{2}} \\ 2\cdot
        d_{\frac{n(G)+1}{2}} &= n(G)-1 \\ d_{\frac{n(G)+1}{2}} &= 
        \frac{n(G)-1}{2}
    \end{align*}
    Therefore, it is proved that if \(G\cong\overline{G}\) and that \(n(G)
    \equiv1\pmod4\) then \(G\) has at least one vertex with degree
    \(\frac{n(G)-1}{2}\).
\end{solution}
\section*{Problem 7}
\subsection*{Section 2.1 Problem 12}
\begin{solution}
    Let \(u,v\) be any two vertices in \(K_{m,n}\). If \(u,v\) are in
    different independent sets then they are adjacent so \(\text{d}(u,v)=1\).
    If they are in the same set then they must have a common neighbor \(w\) in
    the other set so \(\text{d}(u,v)=2\). The diameter is the maximum of the
    distances so we have \(\text{diam}(K_{m,n})=2\). The eccentricity \(
    \epsilon(v)\) is 2 for any vertex \(v\) in \(K_{m,n}\) so the radius is 2,
    hence we get \(\text{diam}(K_{m,n})=\text{rad}(K_{m,n})=2\). Therefore, it
    is proved that the diameter and radius of the bicilique \(K_{m,n}\) is 2.
\end{solution}
\section*{Problem 8}
\subsection*{Section 2.1 Problem 18}
\begin{solution}
    Let \(T\) be a tree on \(n\) vertices with one vertex having maximum
    degrees \(\Delta\) and \(l\) be the number of leaves. Thus there are \(n-l
    -1\) remaining vertices with degrees at least 2. The sum of the degrees in
    \(T\) is
    \begin{align*}
        \sum_{v\in V(T)}\deg(v)=2\cdot e(T)=2(n-1)&\geq 1\cdot\Delta+(
            n-l-1)(2)+l\cdot1\\2n-2&\geq \Delta+2n-2l-2+l\\l&\geq \Delta
    \end{align*}
    Hence, the number of leaves in \(T\) is at least \(\Delta\). Therefore, it
    is proved that every tree with maximum degree \(\Delta>1\) has at least \(
    \Delta\) vertices of degree 1.
\end{solution}
\section*{Problem 9}
\subsection*{Section 2.1 Problem 29}
\begin{solution}
    Suppose that every tree does not have a leaf in its larger partite set or
    one of the sets if they have equal size for the sake of contradiction. Let
    \(T\) be a tree with \(n\) vertices s.t. we have two independent sets \(R
    \) and \(B\) from the vertices of \(T\) with \(|R|\leq|B|\). If there are
    no leaves in \(B\) then there must be at least one leaf in \(R\). Since \(
    |R|+|B|=n\) and \(|R|\leq|B|\), then we have \(|B|\geq\frac{n}{2}\). Since
    \(B\) has no leaves, hence every vertex in \(B\) has degree at least 2 so
    we have \[e(T)=\sum_{v\in B}d(v)\geq2\cdot\frac{n}{2}=n\]However, \(T\) is
    a tree with \(n\) vertices so it has \(n-1\) edges then we have \[e(T)\geq
    n\] but \[e(T)=n-1\not\geq n\] so there is a contradiction. Therefore, it
    is proved that every tree has a leaf in its larger partite set or in both
    if they have equal size.
\end{solution}
\section*{Problem 10}
\subsection*{Section 2.1 Problem 49}
\begin{solution}
    Let \(u,v\) be vertices of \(G\) s.t. \(\textrm{d}(u,v)\geq3\), \(P\) be a
    shortest \(u,v\)-path, \(w\) be the third vertex from \(u\) on \(P\), and
    \(x\) be a vertex not on \(P\). The edges \(\{u,x\},\{w,x\}\) cannot both
    exist simultaneously in \(G\) since otherwise \(P\) will not be the
    shortest \(u,v\)-path. Hence, at least one of \(\{u,x\},\{w,x\}\) does not
    exist in \(G\) so one or both of them exist in \(\overline{G}\). In
    addition, the edges with endpoints \(u\) and each of the other vertices in
    \(P\) are in \(\overline{G}\). We have \(\epsilon(v)\leq2\) for any vertex
    \(v\) in \(\overline{G}\) so \(\textrm{rad}(\overline{G})\leq2\).
    Therefore, it is proved that in a simple graph \(G\), \(\textrm{rad}(G)
    \geq3\implies\textrm{rad}(\overline{G})\leq2\).
\end{solution}
\end{document}